\documentclass[12pt,letterpaper]{article}
\usepackage[utf8]{inputenc}
\usepackage[spanish]{babel}
\usepackage{amsmath}
\usepackage{amsfonts}
\usepackage{amssymb}
\usepackage{graphicx}
\usepackage{anysize}
\usepackage{array}
\usepackage{tabulary}
\newcolumntype{K}[1]{>{\arraybackslash}p{#1}}
\usepackage{framed, color}
\usepackage[table,xcdraw]{xcolor}
\marginsize{3cm}{3cm}{0.5cm}{3cm}  

\usepackage[nottoc,numbib]{tocbibind}
%\usepackage{hyperref}
\newcommand{\grad}{\hspace{-2mm}$\phantom{a}^{\circ}$}
\usepackage{fancyhdr}
\pagestyle{fancy}
\fancyhead[RE, RO]{Universidad Industrial de Santander\\
Escuela de Biología \\
Plan de Trabajo de Grado\\
Modalidad Investigación}
\fancyhead[LE, LO]{\includegraphics[scale=0.13]{images/logouis}}
\fancyfoot[RE, RO]{\scriptsize{Escuela de Biología, Facultad de Ciencias\\
Ciudad Universitaria, Carrera 27 – Calle 9, \\
PBX: (7) 6344000 Ext. 2354 Fax: 6349088-Bucaramanga, Colombia\\ E-mail:biojef@uis.edu.co http://www.uis.edu.co}}
\renewcommand{\headrulewidth}{2pt}
\renewcommand{\footrulewidth}{1pt}
\begin{document}
\renewcommand{\refname}{BIBLIOGRAFÍA}
\
\begin{framed}
\textbf{MODALIDAD: Investigación} (Aclarar si se trata de Investigación Básica, Experimental, Adaptación de Tecnología, etc.).
\end{framed}
\begin{framed}
\textbf{TITULO DEL PROYECTO:}\\
\\
\textbf{Fecha y firma de recibido Escuela de Biología:}\\

\end{framed}

\begin{framed}
\textbf{ALUMNO(S) RESPONSABLE (S):}\\
\textbf{Nombre:}\\
\textbf{Código:}\\
\textbf{Carrera:} Biología
\end{framed}
\begin{flushleft}
\begin{tabular}{|K{7.35cm}|K{7.35cm}|}
\hline

\textbf{DIRECTOR DEL PROYECTO} & \textbf{CODIRECTOR DEL PROYECTO} \\
\textbf{Nombre:} & \textbf{Nombre:} \\
\textbf{E-mail:} & \textbf{E-mail:} \\
\textbf{Máximo titulo académico:} & \textbf{Máximo titulo académico:}\\
\hline
\end{tabular}
\end{flushleft}

\section{RESUMEN EJECUTIVO DEL PROYECTO}
\emph{(Máximo Una Página)}
\section{DESCRIPCIÓN DEL PROBLEMA}
\emph{Contextualice y exponga cuál es el problema o pregunta de investigación a responder con la ejecución del trabajo, basado en la revisión de la literatura efectuada). \textbf{(Máximo 2 páginas)}.}\\

\section{ANTECEDENTES O MARCO TEÓRICO}
\emph{(\textbf{Máximo 4 páginas}, debe incluirse la literatura citada más reciente en el tema a tratar).}

\section{JUSTIFICACIÓN DEL PROYECTO QUE AMERITE SU REALIZACION}
\emph{\textbf{(Máximo una página)}}\\

\section{OBJETIVOS DEL PROYECTO}
\emph{\textbf{(Máximo 1/2 página)}}
\subsection{General:}

\subsection{Específicos}

\subsection{Hipótesis y predicciones}
\emph{\textbf{(Cuando aplique)}}
\section{MATERIALES Y MÉTODOS}
\emph{\textbf{(Máximo tres páginas)}}

\addcontentsline{toc}{section}{BIBLIOGRAFÍA}
\bibliographystyle{abbrv}
\bibliography{%nombre del archivo bibtex
}

\section{RESULTADOS ESPERADOS}
\emph{\textbf{(Máxima 1/2 página)}}\\

\section{PRESUPUESTO TOTAL Y FUENTES DE FINANCIACIÓN}
\emph{\textbf{(Máximo 1/2 página)}}

\section{DURACIÓN Y CRONOGRAMA DE EJECUCIÓN DEL PROYECTO:}
\emph{(esboce en una tabla el cronograma del desarrollo del Plan de Trabajo de grado, \textbf{Máximo 1/2 página})}

\section{INSTITUCIONES}
\emph{ Cuando existan otras instituciones incluidas en el proyecto, mencione el nombre y teléfono de las instituciones involucradas}

\section{REQUERIMIENTOS ADICIONALES}
\emph{(Permisos de colecta, permiso de acceso a material genético, permiso de Comité de Ética, por ejemplo) Debe anexarse una copia de la solicitud o los permisos ya concedidos.}

\newpage

\textbf{Observaciones:} Los firmantes aceptan tener pleno conocimiento de las normas que reglamentan la presentación y evaluación de Trabajos de grado y se comprometen a su cumplimiento.\\
\\
\textbf{Nombre y firma de los Responsables:}
\vspace{1 cm}\\
\noindent\rule{5cm}{0.4pt}\\
Estudiante\\
\vspace{2 pt}\\
\noindent\rule{5cm}{0.4pt}\\
Director\\
\vspace{2 pt}\\
\noindent\rule{5cm}{0.4pt}\\
Codirector\\

\begin{framed}
\textbf{EVALUADORES SUGERIDOS:}
\begin{enumerate}
\item Nombre:\\ 
Institución:\\
Máximo título académico:\\ 
E-mail:\\
Teléfono (celular):\\

\item Nombre:\\
Institución:\\
Máximo título académico:\\
E-mail:\\
Teléfono (celular):\\
\end{enumerate}
\end{framed}

\newpage

\begin{framed}
\textbf{PARA USO EXCLUSIVO DEL COMITÉ DE TRABAJOS DE  GRADO.}\\
En su sesión del día \noindent\rule{5cm}{0.4pt} decide:\\
$\square$ \textbf{Aprobar.} $\square$ \textbf{No aprobar.} $\square$ \textbf{Enviar a evaluación externa.} $\square$ \textbf{Aprobarla, sí realiza las modificaciones.}\\
\textbf{Observaciones:}\\
\noindent{\rule{15cm}{0.4pt}}\\
\noindent{\rule{15cm}{0.4pt}}\\
\noindent{\rule{15cm}{0.4pt}}\\
\noindent{\rule{15cm}{0.4pt}}\\
\end{framed}
\vspace{1 cm}
\noindent{\rule{5cm}{0.4pt}}\\
Presidente del Comité de Trabajos de Grado\\
\vspace{4 pt}\\
\noindent{\rule{5cm}{0.4pt}}\\
Comité de Trabajos de Grado\\
\vspace{4 pt}\\
\noindent{\rule{5cm}{0.4pt}}\\
Comité de Trabajos de Grado\\

\newpage

\begin{center}
\textbf{ACUERDO DE CONFIDENCIALIDAD}
\end{center}
Yo, \noindent\rule{3cm}{0.4pt}, identificado con cedula de ciudadanía No.\noindent\rule{3cm}{0.4pt}, expedida en \noindent\rule{3cm}{0.4pt} y código de estudiante UIS No.\noindent\rule{3cm}{0.4pt}, en mi condición de estudiante de la Universidad Industrial de Santander, me comprometo expresamente con ésta a:

\begin{enumerate}
\item Reconocer que todas las invenciones o innovaciones tecnológicas de procesos, productos e información resultantes de mi actividad o con mi intervención con ocasión de la labor en la que participe para (dependencia o proyecto)  son de propiedad de la Universidad Industrial de Santander, sin perjuicio de los derechos morales que me corresponden como autor o coautor.

\item No divulgar, ni difundir, ni usar, por ningún medio, sin consentimiento escrito de la Universidad Industrial de Santander, la información que conozca o haya conocido desde mi vinculación con el (dependencia o proyecto), desarrolle o resulte de los trabajos que me sean encomendados y que puedan constituir materia de patente, secreto comercial (know-how), modelo de utilidad, diseño industrial o derechos de autor.

\item No adquirir o aprovecharme en beneficios mío o ajeno de las invenciones, informaciones e innovaciones tecnológicas efectuadas por mi, con mi intervención o que conozca durante la vigencia de mi participación o posterior a ella, de propiedad de la Universidad Industrial de Santander.
\end{enumerate}
En consecuencia, manifiesto expresamente conocer la responsabilidad tanto civil y/o penal que se genere por cualquier violación o alguno de los compromisos que adquiero mediante este acuerdo con la Universidad Industrial de Santander, consagrada en:

\begin{enumerate}
\item La Ley 23 de 1982 y Decisión 351 de la Comisión del Acuerdo de Cartagena sobre Derechos de Autor y Derechos Conexos

\item Ley 190 de 1995 o Estatuto Anticorrupción, por incurrir en el delito de Utilización Indebida de Información Privilegiada.

\item Ley 256 de 1996, por incurrir en Actos de Competencia Desleal como la divulgación o explotación, sin autorización de su titular de secretos industriales o cualquier otra clase de secretos empresariales.
\end{enumerate}
En señal de aceptación suscribo el presente documento, en Bucaramanga a los  (\noindent{\rule{1cm}{0.4pt}}) días del mes de \noindent\rule{2cm}{0.4pt} de 2017.\\
Firma: \rule{5cm}{0.4pt}\\
Nombre: \rule{5cm}{0.4pt}\\
Cédula de Ciudadania: \rule{5cm}{0.4pt}\\
\end{document}
